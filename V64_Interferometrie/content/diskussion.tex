\section{Diskussion}
\label{sec:Diskussion}


Bei Betrachtung der gemessenen Kontrastwerte lässt sich die von der Theorie vorhergesagte Filterpositionsabhängigkeit feststellen. 
Auch die Bereiche höchsten Kontrastes sind bei Winkeln von $\phi = \SI{45}{\degree}$ und $\phi = \SI{135}{\degree}$ an den erwarteten Positionen.


Die gemessenen Werte für die jeweiligen Brechungsindcies für Luft und Glas stimmen relativ gut mit den Literaturwerten überein.
Dabei weicht der ermittelte Brechungsindex von Luft mit $n = \num{1.000259}$ um $\Delta = \SI{11.12}{\percent}$ von dem Literaturwert \cite{Dem2} von $n = \num{1.000292}$ ab.
Die Abweichungen des bestimmten Brechungsindexes von Glas mit $n = \num{1.575(032)}$ und dem Literaturwert \cite{Dem2} mit $n = \num{1.45}$ beläuft sich auf $\Delta = \SI{28(7)}{\percent}$.
Damit ist die Bestimmung des Brechungsindexes von Glas ungenauer als die der des Brechungsindexes von Luft. 
Ein Grund könnte die unterschiedliche Methode zur Detektion der Lichtintensität sein. 
Im Fall der Glas-Messung wird nur ein Photodetektor verwendet. 
Im Gegensatz zu der Differenzmessungsmethode wird dort ein vorhandener Untergrund nicht heraussubtrahiert.


Bei der Messung des Brechungsindexes, bei systematisch variiertem Druck in der Luftkammer, kann eine passende Ausgleichsrechnung vollführt werden. 
Damit kann das Lorentz-Lorenz-Gestz in der jeweiligen Näherung idealer Gase als passende Beschreibung für die Druckabhängigkeit des Brechungsindexes bestätigt werden.


%Kurze Zusammenfassung der Ergebnisse
%-Vergleich mit Literaturwerten
%-Vergleich mit verschiedenen Messverfahren
%-bei Abweichungen mögliche Ursachen finden
\section{Auswertung}
\label{sec:Auswertung}
Alle Berechnungen werden mit dem Programm \glqq Numpy" \cite{numpy}, die Unsicherheiten mit dem Modul \glqq Uncertainties" \cite{uncertainties}, die Ausgleichsrechnungen mit dem Modul \glqq Scipy" \cite{scipy} durchgeführt und die grafischen Darstellungen über das Modul \glqq Matplotlib" \cite{matplotlib} erstellt.

\subsection{Kontrastbestimmung}

Die gemessenen Spannungen können über Gleichung \eqref{eq:Kontrast} in die jeweiligen Kontrastwerte überführt werden.
In Abbildung \ref{fig:plot1} sind die so bestimmten Kontrastwerte gegen die Drehwinkel des Polarisationsfilters aufgetragen.


\begin{table}
    \centering
    \caption{Messwerte für die Kontrastbestimmung. }
    \label{tab:kontrast}
    \sisetup{table-format = 1.3}
    \begin{tabular}{S[table-format = 3] S S S }
        \toprule   
        {$\phi \mathbin{/} \si{\degree}$} & {$U_\text{min} \mathbin{/} \si{\volt}$} & {$U_\text{max} \mathbin{/} \si{\volt}$} & {$K$} \\
        \midrule
        0   & 0.923 &  1.163 & 0.12 \\   
        15  & 0.522 &  0.910 & 0.27 \\
        30  & 0.233 &  0.832 & 0.56 \\
        40  & 0.137 &  0.807 & 0.71 \\
        45  & 0.119 &  0.875 & 0.76 \\
        50  & 0.107 &  0.905 & 0.79 \\
        60  & 0.124 &  1.093 & 0.80 \\ 
        75  & 0.328 &  1.205 & 0.57 \\
        90  & 1.021 &  1.246 & 0.10 \\
        105 & 0.802 &  1.900 & 0.41 \\
        120 & 0.338 &  2.900 & 0.79 \\
        130 & 0.164 &  3.260 & 0.90 \\
        135 & 0.155 &  3.170 & 0.91 \\
        140 & 0.181 &  3.270 & 0.90 \\
        150 & 0.347 &  2.920 & 0.79 \\
        165 & 0.757 &  2.040 & 0.46 \\
        180 & 0.860 &  1.112 & 0.13 \\
        \bottomrule

    \end{tabular}
\end{table}

Wie in der Theorie gezeigt, kann aus Gleichung \eqref{eq:Kontrast} der Kontrast in Abhängigkeit des Drehwinkels dargestellt werden.
Damit kann eine Ausgleichsrechnung mit der Funktion aus Gleichung \eqref{eqn:Kon} durchgeführt werden.

\begin{equation}
    \label{eqn:Kon}
    K(\phi) = A \lvert{\cos(\phi)\sin(\phi)}\rvert
\end{equation}

Aus dieser Rechnung lässt sich der Fitparameter zu 
\begin{equation}
    A = \num{1.68(06)}
\end{equation}
bestimmen.


\begin{figure}
    \includegraphics[width=0.9\textwidth]{build/Kontrast.pdf}
    \caption{Messwerte des Kontrastes mit Ausgleichsrechnung gemäß Gleichung \eqref{eqn:Kon}.}
    \label{fig:plot1}
\end{figure}

\subsection{Brechungsindex von Glas}

Die jeweiligen Erfassungen der Nulldurchgänge sind in Tabelle \ref{tab:glas} dargestellt.
Über Gleichung \eqref{eq:n_glas} lässt sich mit diesen Werten der Brechungsindex bestimmen. 


\begin{table}
    \centering
    \caption{Messwerte für die Bestimmung des Brechungsindex von Glas. }
    \label{tab:glas}
    \sisetup{table-format = 2}
    \begin{tabular}{S S S[table-format = 1.8]}
        \toprule
        {\text{Durchgang}} & {$M$} & {$n$} \\
        \midrule
        1     &  34 & 1.54620974  \\
        2     &  34 & 1.54620974  \\
        3     &  34 & 1.54620974  \\
        4     &  35 & 1.57145515  \\
        5     &  37 & 1.62450261  \\
        6     &  34 & 1.54620974  \\
        7     &  36 & 1.59753863  \\
        8     &  36 & 1.59753863  \\
        9     &  34 & 1.54620974  \\
        10    &  37 & 1.62450261  \\
        \bottomrule

    \end{tabular}
\end{table}
Im Mittel kann der Brechungsindex von Glas zu
\begin{equation}
    n_\text{Glas} = \num{1.57466(03174)}
\end{equation}
bestimmt werden.

\subsection{Brechungsindex von Luft}

Die gemessenen Nulldurchgänge ($M_i$) und die über \eqref{eqn:n_luft} bestimmten Brechungsindicies sind der Tabelle \ref{tab:luft} zu entnehmen.

\begin{table}
    \centering
    \caption{Messwerte für die Bestimmung des Brechungsindex von Luft. $M_i$ bezeichnet die Anzahl durchlaufender Interferenzmaxima des $i$-ten Durchganges.}
    \label{tab:luft}
    \sisetup{table-format = 1.8}
    \begin{tabular}{S[table-format = 3.0] S[table-format = 2.0] S S[table-format = 2.0] S S[table-format = 2.0] S}
        \toprule

        {$p \mathbin{/} \si{\milli\bar}$} & {$M_1$} & {$n_1$} & {$M_2$} & {$n_2$} & {$M_3$} & {$n_3$} \\

        \midrule
        0    &  0   &   1.00000000  &  0   & 1.00000000  & 0   &  1.00000000  \\
        50   &  1   &   1.00000633  &  2   & 1.00001266  & 1   &  1.00000633  \\
        100  &  3   &   1.00001899  &  4   & 1.00002532  & 3   &  1.00001899  \\
        150  &  6   &   1.00003798  &  6   & 1.00003798  & 6   &  1.00003798  \\
        200  &  8   &   1.00005064  &  8   & 1.00005064  & 8   &  1.00005064  \\
        250  &  10  &   1.00006330  &  10  & 1.00006330  & 10  &  1.00006330  \\
        300  &  12  &   1.00007596  &  12  & 1.00007596  & 12  &  1.00007596  \\
        350  &  14  &   1.00008862  &  14  & 1.00008862  & 14  &  1.00008862  \\
        400  &  16  &   1.00010128  &  16  & 1.00010128  & 16  &  1.00010128  \\
        450  &  18  &   1.00011394  &  18  & 1.00011394  & 18  &  1.00011394  \\
        500  &  20  &   1.00012660  &  20  & 1.00012660  & 20  &  1.00012660  \\
        550  &  23  &   1.00014559  &  22  & 1.00013926  & 22  &  1.00013926  \\
        600  &  25  &   1.00015825  &  25  & 1.00015825  & 25  &  1.00015825  \\
        650  &  27  &   1.00017091  &  27  & 1.00017091  & 27  &  1.00017091  \\
        700  &  29  &   1.00018357  &  29  & 1.00018357  & 29  &  1.00018357  \\
        750  &  31  &   1.00019623  &  31  & 1.00019623  & 31  &  1.00019623  \\
        800  &  33  &   1.00020889  &  33  & 1.00020889  & 33  &  1.00020889  \\
        850  &  35  &   1.00022155  &  35  & 1.00022155  & 35  &  1.00022155  \\
        900  &  37  &   1.00023421  &  37  & 1.00023421  & 37  &  1.00023421  \\
        950  &  39  &   1.00024687  &  39  & 1.00024687  & 39  &  1.00024687  \\
        990  &  41  &   1.00025953  &  41  & 1.00025953  & 41  &  1.00025953  \\

        \bottomrule

    \end{tabular}
\end{table}

\subsection{Lorentz-Lorenz-Gesetz}

Das in der Theorie \ref{sec:Theorie} beschriebene Lorentz-Lorenz-Gesetz setzt den Brechungsindex und den herrschenden Druck in Zusammenhang.
Über Gleichung \eqref{eqn:lorentz} lässt sich eine Ausgleichsrechnung der Form


\begin{equation}
    n = \sqrt{1+ \frac{a\cdot p}{RT}} + b
\end{equation}

durchführen.
Dazu wird die Temperatur zu $T = \SI{20.7}{\celsius}$ bestimmt. 
Die Ergebnisse sind in Tabelle \ref{tab:lorentz} aufgeführt.

\begin{table}
    \centering
    \caption{Fitparameter der Ausgleichsrechnung für die Brechungsindicies nach dem Lorentz-Lorenz-Gesetz.}
    \label{tab:lorentz}
    \sisetup{table-format = 1}
    \begin{tabular}{S S S[table-format = 1.8]}
        \toprule
        {\text{Messung}} & {$a \mathbin{/} \si{\m^3\per\mol}$} & {$b$}  \\
        \midrule
        1  & \num{1.299(007)} & \num{-3.7(9)e-06} \\
        2  & \num{1.284(006)} & \num{-1.9(7)e-06} \\
        3  & \num{1.298(007)} & \num{-3.9(9)e-06} \\
        \bottomrule

    \end{tabular}
\end{table}


%& \multicolumn{4}{c}{Ohne Haube} & \multicolumn{4}{c}{Mit Haube} \\
%\cmidrule(lr){2-5}\cmidrule(lr){6-9}
%Messwerte: Alle gemessenen physikalischen Größen sind übersichtlich darzustellen.
%
%Auswertung:
%Berechnung der geforderten Endergebnisse
%mit allen Zwischenrechnungen und Fehlerformeln, sodass die Rechnung nachvollziehbar ist.
%Eine kurze Erläuterung der Rechnungen (z.B. verwendete Programme)
%Graphische Darstellung der Ergebnisse
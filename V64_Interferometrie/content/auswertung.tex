\section{Auswertung}
\label{sec:Auswertung}
Alle Berechnungen werden mit dem Programm \glqq Numpy" \cite{numpy}, die Unsicherheiten mit dem Modul \glqq Uncertainties" \cite{uncertainties}, die Ausgleichsrechnungen mit dem Modul \glqq Scipy" \cite{scipy} durchgeführt und die grafischen Darstellungen über das Modul \glqq Matplotlib" \cite{matplotlib} erstellt.


\begin{table}
    \centering
    \caption{Messwerte für die Kontrastbestimmung. }
    \label{tab:kontrast}
    \sisetup{table-format = 1.2}
    \begin{tabular}{S S S S}
        \toprule

        \midrule

        \bottomrule

    \end{tabular}
\end{table}

\begin{equation}
    K = A \lvert{\cos(\Theta)\sin(\Theta)}\rvert
\end{equation}

\begin{equation}
    A = \num{1(1)}
\end{equation}

\begin{figure}
    \includegraphics[width=0.9\textwidth]{build/Kontrast.pdf}
    \caption{Messwerte des Kontrastes mit Ausgleichsrechnung gemäß Gleichung REF.}
    \label{plot1}
\end{figure}

\begin{table}
    \centering
    \caption{Messwerte für die Bestimmung des Brechungsindex von Glas. }
    \label{tab:glas}
    \sisetup{table-format = 1.2}
    \begin{tabular}{S S S}
        \toprule

        \midrule

        \bottomrule

    \end{tabular}
\end{table}


\begin{table}
    \centering
    \caption{Messwerte für die Bestimmung des Brechungsindex von Luft. $M_i$ bezeichnet die Anzahl durchlaufender Interferenzmaxima des $i$-ten Durchganges.}
    \label{tab:luft}
    \sisetup{table-format = 1.2}
    \begin{tabular}{S S S S S S S S S}
        \toprule
        & \multicolumn{4}{c}{Ohne Haube} & \multicolumn{4}{c}{Mit Haube} \\
            \cmidrule(lr){2-5}\cmidrule(lr){6-9}

        {$p \mathbin{/} \si{\milli\bar}$} & {$M_1$} & {$n_1$} & {$M_2$} & {$n_2$} & {$M_3$} & {$n_3$} & {$M_4$} & {$n_4$} \\

        \midrule

        \bottomrule

    \end{tabular}
\end{table}

\begin{equation}
    n = \frac{a}{TR}p+b
\end{equation}

%& \multicolumn{4}{c}{Ohne Haube} & \multicolumn{4}{c}{Mit Haube} \\
%\cmidrule(lr){2-5}\cmidrule(lr){6-9}
%Messwerte: Alle gemessenen physikalischen Größen sind übersichtlich darzustellen.
%
%Auswertung:
%Berechnung der geforderten Endergebnisse
%mit allen Zwischenrechnungen und Fehlerformeln, sodass die Rechnung nachvollziehbar ist.
%Eine kurze Erläuterung der Rechnungen (z.B. verwendete Programme)
%Graphische Darstellung der Ergebnisse
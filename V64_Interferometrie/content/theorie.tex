\section{Theorie}
\label{sec:Theorie}

Licht ist eine elektromagnetische Welle, 
dessen elektrische Komponente im Fall für monochromatischem Licht durch
\begin{equation}
  \label{eq:Welle}
  \vec{E}=\vec{E_0}\exp{\left(i (\omega t-\vec{k}\vec{r})\right)}
\end{equation}
beschrieben werden kann. 
$\vec{E_0}$ gibt die Amplitude und Polarisation des Feldes
an.
$\omega$ ist die Kreisfrequenz und  $\vec{k}$ der Wellenvektor.
Die Intensität einer Welle ist proportional zu der Amplitude $\vec{E}$:
\begin{equation}
  \label{eq:Intensitaet}
  I \propto \langle \lvert \vec{E} \rvert^2 \rangle.
\end{equation} 


\subsection{Kontrast}%%%%%%%%%%%%%%%%%%%%%%%%%%%%%%%%%%%%%%%%%%%%%%%%%%%%%%%%%%%%%%%%%%%%%%%%%%%%%%%%%%%%%%%%%%%%%%%%%%%%%%%%%%%%%%%%%%%%%%%%%%%%%%%%%%%%%%%%%%%%%%%%%%%%%%%%%%%
Der Kontrast des Interferometers wird über 
\begin{equation}
  K(\phi)= \frac{I_{\text{max}}-I_{\text{min}}}{I_{\text{max}}+I_{\text{min}}}
  \label{eq:Kontrast}
\end{equation}
beschrieben, 
wobei $I_{\text{max}} / I_{\text{min}}$ die Intensität der Interferenzmaxima/-minima 
und $\phi$ der Polarisationswinkel ist. 
\newline \newline
\noindent Bei dem Sagnac-Interferometer überlagern sich zwei elektromagnetische Wellen:
\begin{equation}
  E_1 = E_{1} \cos{\omega t} = E_{0,1} \sin(\phi) \cos{(\omega t)} \quad 
  E_2 = E_{2} \cos{(\omega t + \delta) } =E_{0,2} \cos(\phi)  \cos{(\omega t + \delta) } ,
\end{equation}
\noindent sodass für die Intensität mit Gl. \eqref{eq:Intensitaet} folgt.
\begin{align}
  I &\propto \langle \lvert
  E_1+E_2 \rvert^2 \rangle =
  \langle \lvert E_1 \cos{(\phi)}\cos{(\omega t)} + E_2 \sin{(\phi)}\cos{(\omega t + \delta)} \rvert^2 \rangle \\
  &=  \langle E_{0,1}^2 + E_{0,2}^2 + 2E_{0,1}E_{0,2}\cos(\phi)\sin(\phi)\cos{\delta}  \rangle
\end{align}
mit der Phasendifferenz $\delta$ zwischen den beiden Wellen.
$I_{\text{max}}$ tritt bei konstruktiver Interferenz auf,
also für Vielfache von $\delta = 2\pi$.
$I_{\text{min}}$ tritt bei destruktiver Interferenz auf,
also für $\delta = 2\pi(n+1)$ ($n \in \mathbb{Z} $).

Damit ergibt sich für die maximale bzw. minimale Intensität
\begin{equation}
  I_{\text{max/min}} \propto I(1\pm 2\cos{\phi}\sin{\phi}).
  \label{eq:imaxmin}
\end{equation}
\noindent Die Intensitäten werden in Gl. \eqref{eq:Kontrast} eingesetzt:
\begin{equation}
  \label{eqn:K}
  K(\phi)=2\cos{\phi}\sin{\phi}=\sin{2\phi}.
\end{equation}

\subsection{Brechungsindex}%%%%%%%%%%%%%%%%%%%%%%%%%%%%%%%%%%%%%%%%%%%%%%%%%%%%%%%%%%%%%%%%%%%%%%%%%%%%%%%%%%%%%%%%%%%%%%%%%%%%%%%%555
Zur Berechnung des Brechungsindex wird der Phasenversatz $\Delta \Phi$ benötigt.
Dieser lässt sich experimentell über die Anzahl der Interferenzmaxima bestimmen:
\begin{equation}
  M=\frac{\Delta \Phi}{2\pi}
  \label{eq:AnzMax}
\end{equation}

Dies ist möglich, 
da eine Laufzeitdifferenz bwz. eine Phasendifferenz entsteht,
wenn ein Strahl ein Medium mit anderem Brechungsindex durchläuft.
Diese Phasendifferenz führt zu einem veränderten Interzmuster.

\subsubsection{Brechungsindex von Glas}%%%%%%%%%%%%%%%%%%%%%%%%%%%%%%%%%%%%%%%%%%%%%%%%%%%%%
Der Phasenversatz der bei dem Durchlaufen von Glas entsteht,
ist abhängig von der Laufzeitdifferenz, 
also dem Brechungsindex $n$ und der Länge $T$ des durchlaufenden Wegs,
und Phasenverschiebung beim Brechen des Lichtes, das winkelabhängig ist \cite{V64}:

\begin{equation}
\Delta \Phi
(\Theta)=\frac{2\pi}{\lambda_{\text{vac}}}T\Bigl(\frac{n-1}{2n}\Theta^2+\mathcal{O}(\Theta^4)   \Bigr) .
\label{eq:nGlas}
\end{equation}
\noindent wobei $\lambda_{\text{vac}}$ die Vakuumwellenlänge 
und $\Theta$ der Brechungswinkel ist.
Mit Gl. \eqref{eq:AnzMax} ist der Brechungsindex
\begin{equation}
  n = \left(1 - \frac{M \lambda_{\text{vac}}}{2 T \theta_0 \Delta\Phi}\right)^{-1} .
  \label{eq:n_glas}
\end{equation}

\subsubsection{Brechungsindex von Gasen}
Bei dem Durchlaufen eines Gases mit einem anderen Brechungsindex als die umgebende Luft,
entsteht ein Phasenversatz
\begin{equation}
\Delta \Phi(\Theta)=\frac{2\pi}{\lambda_{\text{vac}}} \Delta n L
\label{eq:pM2}
\end{equation}
$L$ ist die Länge des durchlaufenden Weges \cite{V64}.
Darüber lässt sich der Brechungsindex über
\begin{equation}
  \label{eqn:n_luft}
  n = \frac{\lambda_{\text{vac}} M }{L} + 1
\end{equation}
\newline \newline
\noindent Das Lorentz-Lorenz-Gesetz
\begin{equation}
  n \approx \sqrt{1+ \frac{3Ap}{RT}}
  \label{eqn:lorentz}
\end{equation}
gibt den Brechungsindex eines Gases in Anhängigkeit der Temperatur $T$,
dem Druck $p$, dem molaren Brechungsindex $A$ 
und der allgemeinen Gaskonstante $R$ an.

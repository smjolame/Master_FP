\section{Theorie}
\label{sec:Theorie}

The molar heat capacity is defined by 
\begin{equation}
    C = \frac{\Delta Q}{\Delta T}.
\end{equation}

It describes the proportion between the amount of heat $Q$ to raise the tempreture $T$ of a material. 
Regarding the first law of thermodynamics
\begin{equation}
    \symup{d}Q = \symup{d}U+p\symup{d}V,
\end{equation}
the variation of heat is dependend of the pressure $p$ and the volume $V$.
For experimental purposes it is beneficial to keep the pressure or the volume constant.
The difference between these two cases is given by
\begin{equation}
    C_p - C_V = TV \alpha_V^2 B.
\end{equation}
$T$ is the temperature, $V$ the volume, $\alpha_V$ the expansion coefficient and $B$ the bulk modulus.
So this difference depends on properties of the respective material and the state of aggregation. 
Because of the small expansion coefficients of cristalls, the difference between the two heat capacities is neglectable.
For experimental practicability, often the heat capacity by constant pressure is used.


\subsection{Dulong-Petit}
In the classical theory the heat capacity is constant and oly depends on the number of degrees of freedom $f$ of the system:
\begin{equation}
    C = \frac{f}{2} R.
\end{equation} 
For a three dimensional cristall, every atom has three possible directions to oscillate. 
In average the atoms have the same kientic and potential engergy of 
\begin{equation}
    E = \frac{1}{2}k_B T,
\end{equation}
with $k_B$ the Boltzman-constant.
So every atom has six degrees of freedom. 
Eventually the heat capacity results to
\begin{equation}
    C = 3 R,
\end{equation}
where $R$ is the gas constant.
%In knapper Form sind die physikalischen Grundlagen des Versuches, des Messverfahrens, sowie sämtliche für die Auswertung erforderlichen Gleichungen darzustellen. (Keine Herleitung)

%(eventuell die Aufgaben)

%Der Versuchsaufbau: Beschreibung des Versuchs und der Funktionsweise (mit Skizze/Bild/Foto)

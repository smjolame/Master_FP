\section{Theorie}
\label{sec:Theorie}

The molar heat capacity is defined by 
\begin{equation}
    C = \frac{\Delta Q}{\Delta T}.
\end{equation}

It describes the proportion between the amount of heat $Q$ to raise the tempreture $T$ of a material. 
Regarding the first law of thermodynamics
\begin{equation}
    \symup{d}Q = \symup{d}U+p\symup{d}V,
\end{equation}
the variation of heat is dependend of the pressure $p$ and the volume $V$.

%In knapper Form sind die physikalischen Grundlagen des Versuches, des Messverfahrens, sowie sämtliche für die Auswertung erforderlichen Gleichungen darzustellen. (Keine Herleitung)

%(eventuell die Aufgaben)

%Der Versuchsaufbau: Beschreibung des Versuchs und der Funktionsweise (mit Skizze/Bild/Foto)

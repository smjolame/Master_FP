\section{Analysis}
\label{sec:Auswertung}
The following calculations are made with  "Numpy" \cite{numpy}, 
"Uncertainties" \cite{uncertainties}, 
"Scipy" \cite{scipy} and  "Matplotlib" \cite{matplotlib}.

\subsection{Fitting the linear expansion coefficient}
For the following calculations the expansion coefficient is needed.
For that purpose the data in Table \ref{tab:alpha}
is fittet with
\begin{equation*}
    \alpha = a \frac{1}{T} + b:
\end{equation*}
\begin{equation*}
    a = -873 \pm 4
\end{equation*}
\begin{equation*}
    b= 19.411 \pm 0.029
\end{equation*}


\subsection{Calculating the molar heat capacity}
For the calculation of the molar heat capacity 
the current $I$, voltage $U$ , time $t$ and resistance $R$ are measured 
like described in section \ref{sec:Durchführung} (see Table \ref{tab:Messwerte}).
\newline \newline
\noindent The engergy the coppy receives is calculated by 
\begin{equation}
    E = U \cdot I \cdot \Delta t .
\end{equation}
\noindent With \eqref{} and the resistance $R$ the temperature $T$ is determined.
The molar heat capacity for constant pressure ist calculated by
\begin{equation}
    c_p = \frac{ M}{m} \frac{E}{\Delta T}.
\end{equation}
\noindent $M = \SI{63.55}{\g\per\mole}$ is the molar Mass and
$m= \SI{342}{\g}$ \cite{Molmasse_kupfer}\cite{V47}.

\begin{table}
    \centering
    \caption{bla}
    \label{tab:bla}
    \sisetup{table-format = 1.2}
    \begin{tabular}{c c c c}
        \toprule
        $T$ in $\si{\kelvin}$ & $c_v$ in $\si{\joule\per\mole\per\kelvin}$ & $\frac{\Theta_D}{T}$ & $\Theta$ in $\si{\kelvin}$ \\
        \midrule
        103,3  &  16,78  &  4,6  &  475,2  \\
        113,3  &  17,38  &  4,7  &  532,6  \\
        123,4  &  18,50  &  4,9  &  604,6  \\
        133,3  &  19,22  &  3,5  &  466,4  \\
        143,2  &  19,86  &  3,6  &  515,5  \\
        153,1  &  20,36  &  2,0  &  306,3  \\
        163,2  &  21,29  &  2,1  &  342,6  \\
        \bottomrule

    \end{tabular}
\end{table}







%Messwerte: Alle gemessenen physikalischen Größen sind übersichtlich darzustellen.
%
%Auswertung:
%Berechnung der geforderten Endergebnisse
%mit allen Zwischenrechnungen und Fehlerformeln, sodass die Rechnung nachvollziehbar ist.
%Eine kurze Erläuterung der Rechnungen (z.B. verwendete Programme)
%Graphische Darstellung der Ergebnisse
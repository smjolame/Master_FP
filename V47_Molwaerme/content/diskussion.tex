\section{Discussion}
\label{sec:Diskussion}
The heat capacity can be calculated with the with the energy $E$ 
and the temperature difference $\Delta T$.
The heat capacity is expected to not be greater than $3 R$ (see \eqref{})
In figure \ref{fig:cv} one can see that this is true for the most part with the
meassured data.
\newline \newline
\noindent The Debye temperature $\Theta_D$ can be empirically determinted 
and calculated (see Chapter \ref{sub:deb}).
Comparing these to the given value in literature $\Theta_{D,\text{lit}} = 345 \si{\kelvin}$ \cite{deb_kupfer},
one gets the deviations :
\begin{equation*}
    \Theta_{D,\text{empirically}} =(460 \pm 100) \si{\kelvin}
    \quad \Delta =  -0.34 \pm 0.28 = (-34 \pm 28) \%
\end{equation*}
\begin{equation*}
    \Theta_{D,\text{theoretically}} = 332.0 \si{\kelvin}
    \quad \Delta = 0.038 = 3.8\%
\end{equation*}
with
\begin{equation*}
    \Delta = 1- \frac{\Theta_D}{\Theta_{D,\text{lit}} }.
\end{equation*}
\noindent A possible explanation
for the deviation of the empirically determined Debye temperature is, 
that the temperature of the reciepent and the shielding should be the same
to avoid heat radiation loss,
which is not given in this experiment,
because the temperature of the shielding is controlled manually.
Also there could be heat loss caused by convection, 
because the reciepent is not in a perfect vacuum.
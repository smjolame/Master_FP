\section{Diskussion}
\label{sec:Diskussion}
Die Abweichungen vom Literaturwert $m^*_\text{theo} = 0,067 m_e$ \cite{Galliumarsenid_me}
werden über 
\begin{equation*}
   \Delta =  1 - \frac{m^*}{m^*_\text{theo} }
\end{equation*}
\noindent berechnet:
\begin{equation*}
    N = 1,2 \cdot 10^{18} \si{\per\cubic\cm}: 
    \quad m^*=  (0,188 \pm 0,022 )m_e,
    \quad \Delta = (2,74 \pm 0,33) = (274 \pm 33) \%
\end{equation*}
\begin{equation*}
    N = 2,8 \cdot 10^{18} \si{\per\cubic\cm}: 
    \quad m^*= (0,287 \pm 0,034 )m_e,
    \quad \Delta = (4,2 \pm 0,5) = (420 \pm 50) \%
\end{equation*}

\noindent Die errechneten Werte weichen über das doppelte von dem Literaturwert ab.
Ein möglicher Grund hierfür ist, 
dass der Versuch nicht ordentlich justiert ist.
Ein weiterer Grund ist, 
dass das Messen der Winkel nicht genau  erfolgt.
Auf dem Oszillokop ist in einem Winkelbreich nicht zu erkennen wo das Minima genau liegt.
Zudem ,,overloaded'' der Selektivverstärker, was eine genaue Messung weiterhin erschwert.
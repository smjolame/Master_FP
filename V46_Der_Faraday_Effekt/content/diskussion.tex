\section{Diskussion}
\label{sec:Diskussion}
Die Abweichungen vom Literaturwert $m^*_\text{theo} = 0,067 m_e$ \cite{Galliumarsenid_me}
werden über 
\begin{equation*}
   \Delta =  1 - \frac{m^*}{m^*_\text{theo} }
\end{equation*}
\noindent berechnet:
\begin{equation*}
    N = 1,2 \cdot 10^{18} \si{\per\cubic\cm}: 
    \quad m^*=  (0,082 \pm 0,010)m_e,
    \quad \Delta = (-0,23\pm 0,14) = (-23 \pm 14) \%
\end{equation*}
\begin{equation*}
    N = 2,8 \cdot 10^{18} \si{\per\cubic\cm}: 
    \quad m^*= (0,125 \pm 0,015)m_e,
    \quad \Delta = (-0,87 \pm 0,22) = (- 87 \pm 22 ) \%
\end{equation*}

\noindent Die errechneten Werte weichen bis zu 87\% von dem Literaturwert ab.
Ein möglicher Grund hierfür ist, 
dass der Versuch nicht ordentlich justiert ist.
Ein weiterer Grund ist, 
dass das Messen der Winkel nicht genau erfolgt.
Auf dem Oszillokop ist in einem Winkelbreich nicht zu erkennen wo das Minima genau liegt.
Zudem \glqq overloaded\grqq der Selektivverstärker, was eine genaue Messung weiterhin erschwert.
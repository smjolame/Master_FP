\section{Auswertung}
\label{sec:Auswertung}
Alle Berechnungen werden mit dem Programm \glqq Numpy" \cite{numpy}, die Unsicherheiten mit dem Modul \glqq Uncertainties" \cite{uncertainties}, die Ausgleichsrechnungen mit dem Modul \glqq Scipy" \cite{scipy} durchgeführt und die grafischen Darstellungen über das Modul \glqq Matplotlib" \cite{matplotlib} erstellt.
\subsection{Bestimmung der effektiven Masse}
\noindent Wie in dem Abschnitt \ref{subsec:Durchfuehrung} beschrieben,
werden für alle Proben zwei Winkel aufgenommen.
Mit diesen wird die Differenz gebildet 
und durch die Dicke $d$ der Probe geteilt:
\begin{equation*}
    \Theta = \frac{\phi_1 - \phi_2}{2 d}
\end{equation*}
\noindent mit
\begin{equation*}
    d_\text{hochrein} = \qty{5.11}{\mm} ,
    \quad  d_{N=1,2 \cdot 10^{18}}= \qty{1.36}{\mm},
    \quad   d_{N=2,8 \cdot 10^{18}}= \qty{1.296}{\mm}.
\end{equation*}

\noindent $\Theta$ wird gegen $\lambda^2$ geplottet (siehe Abbildung \ref{plot1}).
\begin{figure}
    \includegraphics[width=0.9\textwidth]{build/plot1.pdf}
    \caption{Die Differenz $\Theta$ geplottet gegen das Wellenlängenquadrat $\lambda^2$
    für die hochreine Probe und die n-dortierten Proben 
    ($N = 1,2 \cdot 10^{18} \si{\per\cubic\cm}$ und $N = 2,8 \cdot 10^{18} \si{\per\cubic\cm}$).}
    \label{plot1}
\end{figure}

\FloatBarrier
\noindent Von den Differenzen für die n-dortierten Proben wird $\Theta_\text{hochrein}$ abgezogen
und die Messreihen mit 
\begin{equation}
    \label{ausgleich}
    \Theta = a \lambda^2
\end{equation}

\noindent gefittet (siehe Abbildung \ref{plot2}):
\begin{equation*}
    a_{N=1,2 \cdot 10^{18}} = (5,3 \pm 1,2)\cdot 10^{12} \si{\per\cubic\meter}
\end{equation*}
\begin{equation*}
    a_{N=2,8 \cdot 10^{18}} = (8 \pm 4)\cdot 10^{12} \si{\per\cubic\meter}
\end{equation*}
\FloatBarrier

\begin{figure}
    \includegraphics[width=0.9\textwidth]{build/plot2.pdf}
    \caption{Die Differenz $\Theta$ geplottet gegen das Wellenlängenquadrat $\lambda^2$
    für die n-dortierten Proben mit $\Theta_\text{hochrein}$ korregiert
    und die dazugehörigen Ausgleichsrechnungen.}
    \label{plot2}
\end{figure}

\noindent Aus \eqref{ausgleich} und \eqref{theta} folgt für die effektive Masse
\begin{equation}
    m^* = \sqrt{\frac{e^3_0 \cdot N \cdot B}{8 \pi^2 \epsilon_0 c^3 n \cdot a}}
\end{equation}
\noindent mit $B= \qty{406}{\milli \tesla }$ und $n= 3,57$ \cite{Galliumarsenid_n}:
\begin{equation*}
    m^*_{N=1,2 \cdot 10^{18}} = (7,5 \pm 0,9) \cdot 10^{-32} \si{\kg} 
    =  (0,082 \pm 0,010 )m_e,
\end{equation*}
\begin{equation*}
    m^*_{N=2,8 \cdot 10^{18}} = (1,14 \pm 0,13) \cdot 10^{-31} \si{\kg} 
    =  (0,125 \pm 0,015 )m_e.
\end{equation*}

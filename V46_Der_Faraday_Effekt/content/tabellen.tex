\section{Anhang}
\label{sec:Anhang}

\begin{table}
    \centering
    \caption{Die gemessen Winkel $\phi$ und die errechneten Winkel $\Theta$ für die hochreine Probe.}
    \label{tab:bla}
    \sisetup{table-format = 1.2}
    \begin{tabular}{c c c c}
        \toprule
        $\lambda$ in $\si{\um}$ & $\phi_1$ in Grad  & $\phi_2$ in Grad & $\Theta$ in rad $\si{\per\meter}$ \\
        \midrule
        1,06  &  91,09  &  66,0  &  42,8  \\
        1,29  &  87,05  &  91,0  &  6,7  \\
        1,45  &  102,4  &  86,29  &  27,5  \\
        1,72  &  101,05  &  93,0  &  13,7  \\
        1,96  &  99,5  &  96,1  &  5,8  \\
        2,156  &  143,15  &  134,35  &  15,0  \\
        2,34  &  172,0  &  163,1  &  15,2  \\
        2,51  &  215,0  &  210,0  &  8,5  \\
        2,65  &  252,2  &  246,18  &  10,3  \\       
        \bottomrule
    \end{tabular}
\end{table}

\begin{table}
    \centering
    \caption{Die gemessen Winkel $\phi$ und die errechneten Winkel $\Theta$ für die n-dotierte Probe $N = 1,2 \cdot 10^{18} \si{\per\cubic\cm}$.}
    \label{tab:bla}
    \sisetup{table-format = 1.2}
    \begin{tabular}{c c c c}
        \toprule
        $\lambda$ in $\si{\um}$ & $\phi_1$ in Grad  & $\phi_2$ in Grad & $\Theta$ in rad $\si{\per\meter}$ \\
        \midrule
        1,06  &  261,0  &  250,0  &  70,6  \\
        1,29  &  258,0  &  251,0  &  44,9  \\
        1,45  &  259,0  &  255,0  &  25,7  \\
        1,72  &  276,0  &  274,0  &  12,8  \\
        1,96  &  271,0  &  265,0  &  38,5  \\
        2,156  &  252,0  &  247,0  &  32,1  \\
        2,34  &  232,0  &  224,0  &  51,3  \\
        2,51  &  204,0  &  197,0  &  44,9  \\
        2,65  &  170,0  &  164,0  &  38,5  \\
        \bottomrule
    \end{tabular}
\end{table}

\begin{table}
    \centering
    \caption{Die gemessen Winkel $\phi$ und die errechneten Winkel $\Theta$ für die n-dotierte Probe $N = 2,8 \cdot 10^{18} \si{\per\cubic\cm}$.}
    \label{tab:bla}
    \sisetup{table-format = 1.2}
    \begin{tabular}{c c c c}
        \toprule
        $\lambda$ in $\si{\um}$ & $\phi_1$ in Grad  & $\phi_2$ in Grad & $\Theta$ in rad $\si{\per\meter}$ \\
        \midrule
        1,06  &  91,09  &  66,0  &  168,9  \\
        1,29  &  87,05  &  91,0  &  26,6  \\
        1,45  &  102,4  &  86,29  &  108,5  \\
        1,72  &  101,05  &  93,0  &  54,2  \\
        1,96  &  99,5  &  96,1  &  22,9  \\
        2,156  &  143,15  &  134,35  &  59,3  \\
        2,34  &  172,0  &  163,1  &  59,9  \\
        2,51  &  215,0  &  210,0  &  33,7  \\
        2,65  &  252,2  &  246,18  &  40,5  \\
        \bottomrule
    \end{tabular}
\end{table}

\section{Diskussion}
\label{sec:Diskussion}

\subsection{Erdmagnetfeld}
Die vertikale Komponente des Erdmagnetfeldes wird mit einem Feld von $B_v=\SI{35.25}{\micro\tesla}$ kompensiert. 
Der tatsächliche Wert der vertikalen Komponente beläuft sich auf $B_v=\SI{45.23}{\micro\tesla}$ \cite{mag_DO_2}.
Damit weicht der zur Kompensation verwendete Wert um $\SI{22.07}{\percent}$ von dem genau gemessenen Wert ab.
Die Abweichungen der gemessenen Horizontalkomponenten von $B_{h,1} = \SI{25.1(14)}{\micro\tesla}$ und $B_{h,2} = \SI{24.3(11)}{\micro\tesla}$
vom genau gemessenen Wert $B_h=\SI{19.34}{\micro\tesla}$ belaufen sich auf jeweils $\SI{30(7)}{\percent}$ und $\SI{26(6)}{\percent}$. 
Die Abweichungen liegen alle in einer ähnlichen Größenordnung. 
Der Versuch ist nicht primär auf die Messung des Erdmagnetfeldes ausgelegt. 
Die umgebenen Magnetfelder, welche sich nicht nur aus dem Erdmagnetfeld zusammensetzen, treten hier als Störfelder auf und werden in dem Versuch möglichst genau kompensiert.
Die genaue Kenntnis des Erdmagnetfeldes würde also nicht unmittelbar zur Genauigkeit des Versuches beitragen. 


\subsection{Kernspins}

Aus den g-Faktoren lassen sich die theoretisch bekannten Kernspins bestimmen. 
Deshalb ist die Diskussion der Kernspins direkt mit der Beurteilung der g-Faktoren geknüpft.
Die berechneten Kernspins ergeben sich zu $ I_{87} = \num{1.514(33)}$ und $I_{85} = \num{2.520(25)}$.
Die Erwartungswerte der Kernspins ($ I_{87} = \num{1.5}$, $ I_{87} = \num{2.5}$) liegen innerhalb dieser Unsicherheiten und weichen sehr gering von den bestimmen Werten ab.


\subsection{Isotopenverhältnis}
In dem verwendeten Gasgemisch kommen beide Isotope in einem Verhältnis von 1 : 2 vor. In der Natur liegt das Verhältnis bei 1 : 3, wodurch auf eine Anreicherung mit $^{87}$Rb zu schließen ist. 


\subsection{Periodische Anregung}

Der Quotient aus den $b$-Parametern kann zu $\num{1.69(10)}$ bestimmt werden und weicht damit um $\SI{13(7)}{\percent}$ vom Theoriewert $1.5$ ab.
Für die geringe Anzahl an aufgenommenen Messdaten ist diese Abweichung relativ gering. 
Eine erhöhte Anzahl an Messungen durch eine geringere Schrittweite der RF-Amplituden, besonders im niedrigen Bereich, wird wahrscheinlich die Ausgleichsrechnungen verbessern und damit zu genaueren Ergebnissen führen können.



Der erwartete exponentielle Anstieg des Signals nach erneuter Anregung über das RF-Feld ist klar zu erkennen.

%Kurze Zusammenfassung der Ergebnisse
%-Vergleich mit Literaturwerten
%-Vergleich mit verschiedenen Messverfahren
%-bei Abweichungen mögliche Ursachen finden
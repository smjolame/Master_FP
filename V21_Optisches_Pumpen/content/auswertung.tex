\section{Auswertung}
\label{sec:Auswertung}
Alle Berechnungen werden mit dem Programm \glqq Numpy" \cite{numpy}, die Unsicherheiten mit dem Modul \glqq Uncertainties" \cite{uncertainties}, die Ausgleichsrechnungen mit dem Modul \glqq Scipy" \cite{scipy} durchgeführt und die grafischen Darstellungen über das Modul \glqq Matplotlib" \cite{matplotlib} erstellt.



\subsection{Vertikales B-Feld}

Um die vertikale Komponente des Erdmagnetfeldes zu kompensieren, wird dem ein Gegenfeld einer Flussdichte von $B_v=\SI{35.25}{\mu\tesla}$ entgegengesetzt.


\subsection{B-Feld-Resonanzen}
Die Messung erfolgt gemäß dem in der Durchführung beschriebenen Ablauf. Die jeweiligen Resonanz-Positionen werden über Drehung des Potenziometers erreicht. 
Aus der Anzahl der Drehungen lässt sich der jeweils geflossene Strom bestimmen. 
Dabei entpricht eine Umdrehung für die Sweep- und Vertilalfeld-Spulen einen Strom von $I=\SI{0.1}{\ampere}$ und für die Horizontalfeld-Spule einen Strom von $I=\SI{0.3}{\ampere}$. 
Aus den Strömen lassen sich die von den Helmholtzspulenpaaren erzeugten Magnetfelder über
\begin{equation}
    B_{\text{Helmholtz}} = \frac{8}{\sqrt{125}}\frac{\mu_0 N I}{R}
\end{equation}
berechnen. Die Summe aus Sweep- und Horizontalfeld ergibt die gesamte magnetische Flussdichte.
Die Ströme und die Flussdichten der Resonanzen sind der Tabelle \ref{tab:Resonanz} zu entnehmen. 

\begin{table}
    \centering
    \caption{}
    \label{tab:Resonanz}
    \sisetup{table-format = 1.2}
    \begin{tabular}[h]{S | S S S| S S S|} 
        \toprule
        & \multicolumn{3}{c}{$^{87}$Rb} & \multicolumn{3}{c}{$^{85}$Rb} \\
        \cmidrule(lr){2-4}\cmidrule(lr){5-7}
        $f \mathbin{/} \si{\kilo\hertz}$ & $I_{Sweep} \mathbin{/} \si{\ampere}$ & $I_{Hori.} \mathbin{/} \si{\ampere}$ & $B_{Summe} \mathbin{/} \si{\micro\tesla}$ & $I_{Sweep} \mathbin{/} \si{\ampere}$ & $I_{Hori.} \mathbin{/} \si{\ampere}$ & $B_{Summe} \mathbin{/} \si{\micro\tesla}$ \\
        \midrule
        100     &    0.65 &  0.77  &   39.23   &  0.00 &  0.00  &  46.35    \\
        200     &    0.43 &  0.67  &   52.14   &  0.03 &  0.03  &  66.44    \\
        300     &    0.67 &  1.02  &   66.50   &  0.03 &  0.03  &  87.68    \\
        400     &    0.54 &  0.97  &   85.21   &  0.06 &  0.06  & 111.40    \\
        500     &    0.34 &  0.90  &   99.45   &  0.09 &  0.09  & 133.24    \\
        600     &    0.13 &  0.84  &  113.08   &  0.12 &  0.12  & 155.99    \\
        700     &    0.20 &  1.03  &  122.69   &  0.13 &  0.13  & 172.60    \\
        800     &    0.25 &  1.02  &  141.31   &  0.14 &  0.16  & 198.12    \\
        900     &    0.31 &  0.84  &  155.39   &  0.16 &  0.19  & 219.13    \\
        1000    &    0.16 &  0.82  &  167.21   &  0.18 &  0.22  & 238.91    \\
        \bottomrule
    \end{tabular}
\end{table}
An den Resonanzen muss der Zusammenhang aus Gleichung REF gelten.
Dieser lässt sich unter Berücksichtigung eines Untergrundfeldes zu einer Geradengleichung der Form 
\begin{equation}
    \label{eq:B}
    B(f) = \underbrace{\frac{h}{\mu_B g_F}}_{=a}f + b
\end{equation}
umstellen.

\begin{figure}
    \centering
    \includegraphics[width=\textwidth]{build/B_Felder.pdf}
    \caption{Gemessene B-Felder der Resonanzen in Abhängigkeit der jeweiligen RF-Feld-Frequenzen für beide Rb-Isotope. Anhand der Messwerte lassen sich Ausgleichsgeraden berechnen.}
    \label{fig:Resonanz}
\end{figure}

In grafischer Darstellung (Abbildung \ref{fig:Resonanz}) lässt sich dieser lineare Zusammenhang zwischen der magnetischen Flussdichte und der Frequenz des RF-Feldes erkennen.
Es lässt sich eine Ausgleichsrechnung der Geraden aus Gleichung \eqref{eq:B} durchführen.
Die Ergebnisse dieser Rechnung ergeben sich für $^{87}$Rb zu  
\begin{align*}
    a_{87} = & \SI{0.1439(23)}{\micro\tesla\per\kilo\hertz} \\
    b_{87} = &\SI{25.1(14)}{\micro\tesla} 
\end{align*}
und für $^{85}$Rb zu
\begin{align*}
    a_{85} =& \SI{0.2158(18)}{\micro\tesla\per\kilo\hertz} \\
    b_{85} =&  \SI{24.3(11)}{\micro\tesla}.
\end{align*}
Der konstante Untergrund lässt sich hauptsächlich auf die verbleibende horizontalkomponente des Erdmagnetfeldes zurückführen. 



Aus den Steigungen der Ausgleichsgeraden lässt sich der jeweilige g-Faktor bestimmen.
Dazu wird die Steigung $a$ aus Gleichung \eqref{eq:B} nach dem g-Faktor $g_F$ umgeformt.
So lässt sich über
\begin{equation}
    g_F = \frac{h}{\mu_B a}
\end{equation}
der g-Faktor für $^{87}$Rb zu 
\begin{align*}
    g_F = \num{0.496(8)}
\end{align*}
und für $^{85}$Rb zu 
\begin{align*}
    g_F=\num{0.3311(27)}
\end{align*}
bestimmen.



%Messwerte: Alle gemessenen physikalischen Größen sind übersichtlich darzustellen.
%
%Auswertung:
%Berechnung der geforderten Endergebnisse
%mit allen Zwischenrechnungen und Fehlerformeln, sodass die Rechnung nachvollziehbar ist.
%Eine kurze Erläuterung der Rechnungen (z.B. verwendete Programme)
%Graphische Darstellung der Ergebnisse
\section{Diskussion}
\label{sec:Diskussion}

\subsection*{Invertierender-Linearverstärker}
Zur Untersuchung des invertierenden Linearverstärkers 
werden die Ausgangsspannung und die Phase zwisschen Ein- und Ausgang in Abhängigkeit der Frequenz gemessen.
Die Verstärkung wird doppeltlogarithmisch gegen die Frequnz geplottet (siehe Abbildung \ref{fig:int100}).
Es ist erkennbar, dass diese konstant für einen kleinen Frequenzbrecih ist
und ab einer Frequenz mit steigender Frequent abnimmt.
Durch einen Fit durch das Plateau lässt sich die Leerlaufverstärkung bestimmen.
Werden die Ergebnisse des Fits und dem Theoriewert aus der Tabelle \ref{tab:Leerlaufverstarkung}
ist eine relative Abweichungen von


\begin{table}
    \centering
    \sisetup{table-format = 1.2}
    \begin{tabular}{c c }
        \toprule
        $R_2$ in $\qty{}{\kilo\ohm}$ & $\Delta V_0 = 1 - \frac{V_0}{V_\text{0,ideal}}$ \\
        \midrule   
        100 &   0.05755395683453235 \pm 0.00000000000000004     \\
        150 &   0.061 \pm 0.006     \\
        47  &   0.038 \pm 0.0026        \\
        \bottomrule   
    \end{tabular}
\end{table}
festzustellen.
Ein möglicher Grund für diese Abweichung ist, 
dass die verwendeten Wiederstände nicht genau ihrem angegeben entsprechen.
\newline \newline
\noindent Dass die Verstärkung mit steigender Frequenz abnimmt,
entspricht dem erwartenden Verlauf,
da hier nicht von einem idealen Operationsverstärker ausgegangen wird.
\newline \newline
\noindent In der Abbildung \ref{fig:int100f} ist erkennbar,
dass die Phasenverschiebung in einem Frequenzbrecih konstant bei 180 ist, 
jedoch auch mit steigender Frequenz abfällt.
Auch dieses Verhalten lässt sich dadurch erklären, 
dass der verwendete Operationsverstärker nicht ideal ist, 
also eine frequenzabhängig aufweist.

\subsection{Umkehr-Integrator}
Die Verstärkung des Umkehr-Integrator wird frequenzabhängig aufgenommen 
und doppeltlogarithmisch geplottet (siehe Abbildung \ref{fig:umint}).
In der doppeltlogarithmisch Darstellung ist ein linearer Verlauf zu sehen.
Dies stimmt mit den theoretischen Erwartungen überein.
\newline \newline

Weiterhin ist zu erkennen (siehe Abbildung \ref{fig:int_os}), 
dass der Integrator das Eingangssignal erfolgreich integriert.
Aus einer Sinusspannung wird eine Cosinusspannung 
und aus einer Rechteckspannung eine Dreieckspannung.

\subsection{Invertierender Differenzierer}
Auch bei dem invertierenden Differenzierers ist in der doppeltlogarithmischen
Darstellung ein linearer Verlauf zu erkennen (siehe Abbildung \ref{fig:dif})
\newline \newline
\noindent Aus das Eingangssignal wird erfolgreich differenziert.
Das Sinussignal wird zum Cosinus, 
das Rechtecksignal zu Delta-Peaks und die Dreieckspannung zu einer Rechteckspannung.

\subsection{Nicht invertierender Schmitt-Trigger}
Der nicht invertierenden Schmitt-Triggers macht aus einem Dreiecksignal ein Rechtecksignal
(siehe Abbildung \ref{fig:schmitt}).
 Die dabei ermittelte Kippspannung 
$U_K = \qty{-1.37}{\V}$ weicht von der berechneten Kippspannung $U_\text{K,theo} = \qty{-1.448}{\V}$
um 5,4\% ab. 
Die experiementell bestimmten Kippspannung stimmt mit der Theorie weitesgehend übereien.

\subsection{Signalgenerator}
Der Signalgenerator generiert die Dreieckspannung $U_A$ und 
die Rechteckspannung $U_1$.
Dieses Verhalten kann in der Abbildung \ref{fig:sig} beobachtet werden.
Die Frequenz und die Amplitude der Dreieckspannung wird bestimmt und theoretisch berechnet
(siehe Tabelle \ref{tab:sig}).
Es werden Abweichungen von $\Delta f = 0,32$ und $\Delta A = -1$.
Mögliche Gründe dafür sind,
dass die Material Eigenschaftschen nicht genau angegeben sind
oder dass das Signalrauschen die ermmitelten Werte verschlechtern.

\section{Durchführung}
\label{sec:Durchführung}

Im Rahmen des Versuches werden verschieden Schaltungen verwendet, in denen die arbeits- und funktionsweise von Operationsverstärkern untersucht werden kann.
Alle auftretenen Schaltungen sind in der Theorie \ref{sec:Theorie} erläutert.
Diese werden auf einer Steckplatine verschaltet.
Ein Funktionsgenerator erzeugt die Einganssignale.
Die Ein- und Ausgangssignale werden über ein Oszilloskop dargestellt und ausgewertet. 
Der Operationsverstärker wird mit einer Versorgungsspannung von jeweils $U_B=\pm\SI{15}{\volt}$ versorgt.




\subsection{Invertierender-Linearverstärker}
Zuerst wird die Schaltung eines invertierenden Linearverstärkers (\ref{fig:Inv_Lin}) aufgebaut. 
Das sinusförmige Einganssignal besitzt eine Schwingungsbreite von $U_e^{PP}=\SI{139}{\milli\volt}$.
Bei gegebenen Widerstandswerten wird systematisch die Eingangsfrequenz erhöht.
Dabei wird die jeweilige Ausgangsspannung und die Phase zwisschen Ein- und Ausgang vermessen.
Diese Messung wird für drei verschiedene Paare von Widerstandswerten wiederholt.
Bei gleichbleibenden Widerstand $R_1=\SI{1}{\kilo\ohm}$ werden für den zweiten Widerstand die Werte $R_{2,i}=\{\SI{100}{\kilo\ohm},\SI{150}{\kilo\ohm},\SI{47}{\kilo\ohm}\}$ verwendet.

\subsection{Umkehr-Integrator}

Die Schaltung wird zu einem Umkehr-Integrator (\ref{fig:Um_Int}) umgebaut. 



%Was wurde gemessen bzw. welche Größen wurden variiert?
\section{Durchführung}
\label{sec:Durchführung}

Im Rahmen des Versuches werden verschieden Schaltungen verwendet, in denen die arbeits- und funktionsweise von Operationsverstärkern untersucht werden kann.
Alle auftretenen Schaltungen sind in der Theorie \ref{sec:Theorie} erläutert.
Diese werden auf einer Steckplatine verschaltet.
Ein Funktionsgenerator erzeugt die Einganssignale.
Die Ein- und Ausgangssignale werden über ein Oszilloskop dargestellt und in Form von Aufnahmen für die spätere Auswertung abgespeichert. 
Der Operationsverstärker wird mit einer Versorgungsspannung von jeweils $U_B=\pm\SI{15}{\volt}$ versorgt.




\subsection{Invertierender-Linearverstärker}
\label{sub:Intlin}
Zuerst wird die Schaltung eines invertierenden Linearverstärkers (\ref{fig:Inv_Lin}) aufgebaut. 
Das sinusförmige Einganssignal besitzt eine Schwingungsbreite von $U_e^{PP}=\SI{139}{\milli\volt}$.
Bei gegebenen Widerstandswerten wird systematisch die Eingangsfrequenz erhöht.
Dabei wird die jeweilige Ausgangsspannung und die Phase zwisschen Ein- und Ausgang vermessen.
Diese Messung wird für drei verschiedene Paare von Widerstandswerten wiederholt.
Bei gleichbleibenden Widerstand $R_1=\SI{1}{\kilo\ohm}$ werden für den zweiten Widerstand die Werte $R_{2,i}=\{\SI{100}{\kilo\ohm},\SI{150}{\kilo\ohm},\SI{47}{\kilo\ohm}\}$ verwendet.

\subsection{Umkehr-Integrator}
\label{sub:Um_Int}

Die Schaltung wird zu einem Umkehr-Integrator (\ref{fig:Um_Int}) umgebaut. 
Dabei wird ein Widerstand mit $R = \SI{10}{\kilo\ohm}$ und ein Kondensator mit $C=\SI{100}{\nano\farad}$ verwendet.
Als Einganssignal wird wieder ein sinusförmiges Signal angelegt.
Bei systematischer Änderung der Eingangsfrequenz wird in Abhängigkeit dieser die Ein- und Ausgangsspannung vermessen.
Darüberhinaus werden Aufnahmen des Oszilloskop für sinusförmige, dreieckförmige und rechteckförmige Einganssignale angefertigt.



\subsection{Invertierender Differenzierer}
Als Nächstes wird die Schaltung eines invertierenden Differenzierers (\ref{fig:Inv_Dif}) aufgebaut.
Dazu wird ein Widerstand mit $R=\SI{100}{\kilo\ohm}$ und ein Kondensator mit $C=\SI{22}{\nano\farad}$ eingefügt.
Die Messungen und die Aufnahmen des Oszilloskop werden identisch zu dem Umkehr-Integrator im vorherigen Abschnitt \ref{sub:Um_Int} durchgeführt.


\subsection{Nicht invertierender Schmitt-Trigger}
\label{sub:schmitt}
Es wird die Schaltung eines nicht invertierenden Schmitt-Triggers gemäß Abbildung \ref{fig:Schmitt} verwendet.
Die eingefügten Widerstandswerte werden zu $R_1=\SI{10}{\kilo\ohm}$ und $R_2=\SI{100}{\kilo\ohm}$ gewählt.
Ein dreieckförmiges Signal wird als Einganssignal verwendet.
Auf diese Weise lässt sich die Schaltereigenschaft der Schaltung untersuchen.
Sobald das Einganssignal den Scheitelwert überschreitet, kippt die Ausgangsspannung.
Zur Auswertung wird eine Aufnahme von dem Oszilloskop angefertigt, welches Ein- und Ausgangssignale anzeigt.


\subsection{Signalgenerator}
\label{sub:sig}
Der Schmitt-Trigger aus dem vorherigen Abschnitt wird mit der Schaltung eines invertierenden Integrators zu einem Signalgenerator gemäß Abbildung \ref{fig:Signal} erweitert.
Die Bauteile haben Widerstands- und Kapazitätswerte von $R_1=\SI{10}{\kilo\ohm}$, $R_2=\SI{100}{\kilo\ohm}$, $R_3=\SI{1}{\kilo\ohm}$ und $C=\SI{1}{\micro\farad}$.
Es wird die Frequenz und die Amplitude der sich spontan einstellenden Schwingung vermessen.
Auch hier wird die Anzeige des Oszilloskops gespeichert und für die weitere Auswertung verwendet.

%Was wurde gemessen bzw. welche Größen wurden variiert?